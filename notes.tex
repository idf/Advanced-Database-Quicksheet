%&latex
\documentclass{article}
\usepackage{commons/commons}
\usepackage{commons/note_style}

\usepackage{amssymb}

\def\ojoin{\setbox0=\hbox{$\bowtie$}%
  \rule[-.02ex]{.25em}{.4pt}\llap{\rule[\ht0]{.25em}{.4pt}}}
\def\leftouterjoin{\mathbin{\ojoin\mkern-5.8mu\bowtie}}
\def\rightouterjoin{\mathbin{\bowtie\mkern-5.8mu\ojoin}}
\def\fullouterjoin{\mathbin{\ojoin\mkern-5.8mu\bowtie\mkern-5.8mu\ojoin}}

\begin{document}

%+Title
\title{CSE 232A Database}
\author{Daniel D. Zhang}
\date{Fall 2015}
\maketitle
%-Title

%+Abstract
%\begin{abstract}
%This is the notes for CSE 232A.
%\end{abstract}
%-Abstract

%+Contents
\tableofcontents
%-Contents
\section{Query Processing}
Notations:
\begin{enumerate}
\item $\sigma$: Filter, WHERE (conditions)
\item $\times$: Cartesian, FROM product 
\item $\Pi$: Projection, SELECT (attributes)
\item $x\ra y$: Rename $x$ to $y$. Scalar func: $a+b\ra y$, String ops: $c||d \ra y$

Scalar functions: their input comes from the same tuple (as opp. aggregate)
$$
\Pi_{x, y\ra z, x+y\ra w} R
$$
$PLUS_{x, y\ra z}$, $CONCAT_{x,y\ra z}$
\item $\bowtie$: Natural join: $R\bowtiewS = \Pi_A\sigma_C(R\times S)$. Notice that the $\times$ is \textbf{bag} version. 
\item $\bowtie_{\theta}$: $\triangleq \sigma_{\theta} (R\times S)$. $\theta$ is a condition. 
\end{enumerate}
\subsection{Query Plan}
A simple SFW (Select From Where). 
\begin{enumerate}
\item Naive plan:
$$
\Pi_{attr_1,attr_2}^{FLY}\Big[\sigma_{cond_1\wedge cond_2}^{FLY} (R^{SCAN}\times S^{SCAN})\Big]
$$

FLY and SCAN are how exactly the plan is run. 

\item Join Plan:
$$
\Pi_{attrs} \Big[(\sigma_{cond_1}R)  \bowtie^{HASH} (\sigma_{cond_2}S) \Big]
$$
\item Index plan:

$$
\Pi_{attrs}\sigma_{cond_2}(\sigma_{cond_1}^{INDEX} \bowtie^{RI} S)
$$
$\bowtie^{RI}$ Right Index Join. 

\end{enumerate}
Complex
\begin{itemize}
\item Nested SQL algebra: see notes for rewriting. Essentially, convert nested SQL to JOIN.
\end{itemize}

\subsection{Bag}
\begin{enumerate}
\item Algebraic Operators (Bag Version)
$$
R\cup S
$$
\item Convert to set (DISTINCT)
$$
\delta(R)
$$
\end{enumerate}

%+Bibliography
\begin{thebibliography}{99}
\bibitem{Label1} ...
\bibitem{Label2} ...
\end{thebibliography}
%-Bibliography
\end{document}
